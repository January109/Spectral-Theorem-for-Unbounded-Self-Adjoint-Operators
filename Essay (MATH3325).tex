\documentclass[10pt]{amsart}
\usepackage{styles} 
\title{The spectral theorem for general self-adjoint operators}
\begin{document}
\maketitle
\section{Background}
% We want to find a decomposition for general 
% We look to establish a spectral decomposition for general self-adjoint operators, analogous to the diagonal decomposition of a compact self-adjoint operator. 

% For a compact, self-adjoint operator $T$ on a Hilbert space $H$, we have a 

For a compact, self-adjoint operator $T$ on a Hilbert space $H$, we have a decomposition into orthogonal projections $T = \sum_{n \in \N}\lambda_n P_{E_n}$ for mutually orthogonal finite-dimensional subspaces $E_n$, and in this essay we describe a generalisation to a much larger class of operators, and the theory behind its proof. 

Unlike in the compact case, we cannot guarantee existence of eigenvalues, or that such an operator splits into countably many orthogonal projections up to scale, and in the generalisation we replace this countable sum with an integral against a suitable notion of a projection-valued measure $E$ on a subset of $\C$, so that $T$ has a ``diagonal decomposition'' $\int \lambda dE$. The proof relies heavily on a correspondence between general self-adjoint operators and a certain class of unitary operators, which corresponds to a map $\R \to S^1 \setminus \set1$. We combine this with a way to convert measurable functions into operators related to $T$, alongside the spectral theorem for normal, bounded operators.

We work mostly with the complex case, as we have more tools to work with in this case, and briefly discuss how the real version of this theorem follows. We then give an application to semigroups generated by self-adjoint operators.

% Key points: For compact self-adjoint, eigenvalue/orthogonal projection decomposition; Develop functional calculi, tool for converting measurable functions on spectrum into operators; polynomials become polynomials in $T$; Cayley transform gives correspondence with (bounded) unitary operators, so will rely quite heavily on bounded normal case.

% Introduction: Want an analogous decomposition to compact separable. Will focus on complex mostly, and discuss real at end. Develop functional calculi for operators. Proof involves spectral theorem for normal operators quite heavily.

% Notation: $B(H)$ for bounded operators, $H$ Hilbert space, $\Borel(X)$ for Borel $\sigma$-algebra. 

% Denote the set of complex-measurable functions on a space $Y$ by $\M(Y)$, and the bounded complex-measurable functions by $\M_\infty(Y)$. 
\section{Unbounded Operators}
An \textbf{unbounded operator $T$} on a Hilbert space $H$ is a linear map defined on a linear subspace $D(T)$ of $H$, and taking values in $H$. In this case, $D(T)$ is the \textbf{domain} of $T$, and we write $N(T) \defeq \set{x \in D(T) \mid Tx = 0}$, $R(T) \defeq \set{Tx \mid x \in D(T)}$ and $G(T) \defeq \set{(x, Tx) \in H \times H \mid x \in D(T)}$ for the \textbf{null space}, \textbf{range} and \textbf{graph of $T$} respectively. We say that $T$ \textbf{extends} $S$ if $D(S) \subseteq D(T)$ and $T|_{D(S)} = S$, and write $S \subset T$ for this. We say $T$ is \textbf{densely defined} if $\ol{D(T)} = H$, and \textbf{closed} if $G(T) \subseteq H \times H$ is closed. If $T$ is injective, its inverse $T^{-1}$ has $D(T^{-1}) = R(T)$, and sends $Tx$ to $x$; and if $S$ is another operator, $S + T$ and $ST$ are the operators with domains $D(S + T) = D(S) \cap D(T)$ and $D(ST) = \set{x \in D(T) \mid Tx \in D(S)}$, and $(S + T)(x) \defeq S(x) + T(x)$, $(ST)(x) = S(T(x))$.

Throughout the remaining sections, unless otherwise specified, $H$ is understood to be a \emph{complex} Hilbert space, and operators understood to be unbounded and on $H$. % Change - any Hilbert space here, complex laterstill re
\begin{lemma}
    Let $T$ be a densely defined operator. Then
    \begin{enumerate}
        \item $D(T^*) = \set{y \mid x \mapsto (\lrangle{Tx, y}) \in D(T)^*} \subseteq H$ is a subspace;
        \item For each $y \in D(T^*)$ there is unique $T^*y \in H$ with $\lrangle{Tx, y} = \lrangle{x, T^*y}$ for all $x \in D(T)$, and $T^*$ is linear.
    \end{enumerate}
\end{lemma}
\begin{proof}
    For each $y \in D(T^*)$, by Hahn-Banach and density, the functional $x \mapsto \lrangle{Tx, y}$ extends uniquely to a functional on $H$, and the Riesz representation theorem gives unique $T^*y \in H$ with the desired property.
\end{proof}
\begin{definition}
    Let $T$ be a densely defined operator. Its \textbf{adjoint} is the operator $T^*$ as defined above, and $T$ is \textbf{self-adjoint} if $T = T^*$.
\end{definition}
Note that $\lrangle{Tx, y} = \lrangle{x, T^*y}$ only implies $T \subset T^*$, so being self-adjoint also requires $D(T^*) \subseteq D(T)$.
% Lemma 19.3
It is also clear from the above definitions that if $S$ is densely defined and $S \subset T$, then $T^* \subset S^*$. We will need the following properties of unbounded operators.
\begin{proposition}[19.5, 19.8 in \hyperlink{meisevogt}{[1]}]\label{adjclosed}
    Let $S$ and $T$ be operators. 
    \begin{enumerate}
        \item If $T$ be a densely defined operator, then $T^*$ is closed, and $N(T^*) = R(T)^\perp$;
        \item if $S$ is closed and $T \in B(H)$, then $S + T$ and $S^{-1}$ are closed;
        \item if $S + T$ is densely defined, then $S^* + T^* \subset (S + T)^*$, with equality if $T \in B(H)$;
        \item if $T$ is self-adjoint, $\sigma(T)$ is a non-empty subset of $\R$.
    \end{enumerate}
\end{proposition}
% give commentary on spectra
To generalise spectra to unbounded operators, the resolvent also requires having image $H$, on top of being bijective.
\begin{definition}
    Let $H$ be a Hilbert space over $F = \R$ or $\C$, and $T$ be an operator on $H$. The \textbf{resolvent set} of $T$ is $\rho(T) \defeq \set{z \in F \mid N(zI - T) = 0, R(zI - T) = H}$, and the \textbf{spectrum} of $T$ is $\sigma(T) \defeq F \setminus \rho(T)$.
\end{definition}
Although we have given the general definition, we will almost always work over complex Hilbert spaces, except for a brief discussion of the spectral theorem over a real Hilbert space. The spectrum of a closed operator has the following properties.
\begin{proposition}\label{closedprops}
    Let $T$ be a closed operator. Then $(zI - T)^{-1} \in B(H)$ for every $z \in \rho(T)$; $\sigma(T)$ is closed.
\end{proposition}
\begin{proof}
    For $z \in \rho(T)$, $(zI - T)^{-1}$ is closed with $D((zI - T)^{-1}) = H$, so is continuous by the closed graph theorem. If $z_0 \in \rho(T)$, we use that for $\abs{z - z_0}\norm{(z_0I - T)^{-1}} < 1$, $(z - z_0)(z_0I - T)^{-1} + I \in B(H)$ is invertible.
\end{proof}
% closed implies $(zI - T)^{-1}$ bounded for all $z \in \rho(T)$.
In proving the spectral theorem for unbounded self-adjoint operators, we will use the following transform to reduce to the bounded, normal case. The above result shows that $\pm iI - T$ is injective with domain $D(T)$ and range $H$ when $T$ is self-adjoint, so the operator defined below is well-defined.
\begin{definition}
    Let $T$ be self-adjoint. The operator $U = \U(T) \defeq (iI - T)(-iI - T)^{-1} \in B(H)$ is the \textbf{Cayley transform} of $T$.
\end{definition}
We are interested in the Cayley transform as it yields the following correspondence between unbounded self-adjoint operators and unitary operators on $H$, which will be essential to our proof of the spectral theorem.
\begin{proposition}\label{corresp}
    There is a one-to-one correspondence between unbounded self-adjoint operators $T$ and unitary operators $U \in B(H)$ with $I - U$ injective, sending a self-adjoint operator $T$ to its Cayley transform $\mathbf U = \mathbf U(T) \defeq (iI - T)(-iI - T)^{-1}$, and such a unitary operator $U$ to $\mathbf T = \mathbf T(U) \defeq i(I + U)(I - U)^{-1}$, with $D(\mathbf T) = R(I - U)$.
\end{proposition}
\begin{proof}
    For $T$ self-adjoint, we check $\U(T)$ is unitary and $I - \U(T)$ is injective by direct computation. To show $\T \defeq \T(U)$ is self-adjoint, $\T$ is densely defined as $R(I - U)^\perp = N((I - U)^*) = N(I - U^{-1}) = N(I - U) = \set{0}$. We can show $\lrangle{\T(x), y} = \lrangle{x, \T(y)}$ directly, and for $y \in D(\T^*)$, $z = \T^*y$, we have $U(z + iy) = z - iy$, so $y = (I - U)\brac{\frac1{2i}(z + iy)} \in R(I - U) = D(\T)$. Checking that $\U$ and $\T$ are mutual inverses is again a direct computation.
\end{proof}
% 20.2. (+ sa implies closed), 20.3; 20.5 (one-line), 20.6 + 20.7
% Unbounded / Symmetric operators (19.1 - 19.11) -- One-line sketches (key identities).
% Resolutions of the identity
In lieu of this correspondence, we state a result on the spectrum of a unitary operator, which we will need later.
\begin{proposition}[17.6, 17.17 in \hyperlink{meisevogt}{[1]}]
    Let $U \in B(H)$ be unitary. Then $\sigma(U)$ is a non-empty subset of $S^1$.
\end{proposition}
\section{The Functional Calculus}
A \textbf{$\mathbf{C^*}$-algebra} $X$ is an associative $\C$-algebra with unit $e$ and a Banach space, such that $\norm{xy} \leq \norm{x}\norm{y}$ for all $x$ and $y$, with a $\C$-antilinear map $* : X \to X$ such that $x^{**} = x$, $(xy)^* = y^*x^*$, and $\norm{x^*x} = \norm{x^*}\norm{x}$. If $X$ and $Y$ are $C^*$-algebras, $\varphi \in B(X, Y)$ is a \textbf{$\mathbf{*}$-homomorphism} if $\varphi(xx') = \varphi(x)\varphi(x')$ and $\varphi(x^*) = \varphi(x)^*$ for all $x, x' \in X$. %, and a \textbf{$\mathbf{*}$-isomorphism} if it is invertible, and its inverse is a $*$-homomorphism. % For $a \in A$, we say $a$ is \textbf{normal} if $a^*a = aa^*$, and \textbf{invertible} if there is $b \in A$ with $ab = ba = e$. We define its \textbf{spectrum} as $\sigma(a) = \set{x \in \C \mid xa - e \text{ is invertible}}$.
% $C^*$-algebra; $*$-homomorphism/isomorphism
If $X \subseteq \C$ is compact, we say a map $\Psi : \M_\infty(X) \to B(H)$ is \textbf{weakly-continuous} if for any $x, y \in H$ and $(f_n)_{n \in \N}$ uniformly bounded and pointwise convergent to $f$, we have $\lrangle{\Psi(f_n)x, y} \to \lrangle{\Psi(f)x, y}$.

For maps between other spaces, weak continuity is understood to be the usual notion of continuity, with weak topologies in place of norm topologies. We look to define the functional calculus of an operator, which associates measurable functions to operators on $H$ in a way compatible with $T$. 

In the remaining sections, for $X \subseteq \C$, $\Borel(X)$ denotes the Borel $\sigma$-algebra on $X$, $\M(X)$ and $\M_\infty(X)$ denote measurable and bounded measurable functions $f : X \to \C$ respectively. For $X \subseteq \C$ compact, $\M_\infty(X)$ is a commutative $C^*$-algebra, with pointwise operations, $\norm{\cdot}_\infty$, and $f^* = \ol f$ the pointwise conjugate. $B(H)$ is also a $C^*$-algebra, with the operator norm and composition, and $T*$ the adjoint of $T$.
\begin{proposition}[18.3, 18.4 in \hyperlink{meisevogt}{[1]}]\label{funccalc}
    If $T \in B(H)$ is normal, then there is a unique weakly-continuous $*$-homomorphism $\Psi : \M_\infty(\sigma(T)) \to B(H)$ with $\Psi(z) = T$. Further, 
    \begin{enumerate}
        \item $\Psi$ is continuous with $\norm{\Psi} = 1$;
        \item $R(\Psi)$ consists of normal operators which commute with one another, and $\Psi(f)$ is self-adjoint for $f$ real-valued;
        \item $\Psi(\chi_M)$ is orthogonal projection for each $M \in \Borel(\sigma(T))$.
    \end{enumerate}
\end{proposition}
\begin{definition}
    The \textbf{functional calculus} of a normal operator $T$ is the map $\Psi : \M_\infty(\sigma(T)) \to B(H)$ as above.
\end{definition}
% Additional background: $T$ unitary iff $\sigma(T) \subseteq S^1$. Spectral radius somewhere ($r(A) = \norm{A}$ if $A \in B(H)$ is normal)

% Functional calculi -- Proposition 17.21 (give explicitly)

% [[Give commentary]]
% \begin{proposition}[17.21 in \textbf{Meise-Vogt}]
%     Let $A$ be a $C^*$-algebra, and $a \in A$ be normal. Then there is a unique isometric $*$-homomorphism $\Phi : C(\sigma(a)) \to A$ with $\Phi(z) = a$.
% \end{proposition}
% prop 18.3

% spectral radius formula (normal case)

\section{Spectral Measures}
\begin{definition}
    Let $F = \R$ or $\C$, $Y \subseteq F$, and $H$ be a Hilbert space over $F$. A \textbf{spectral measure} on $Y$ is a map $E : \Borel(Y) \to B(H)$ such that
    \begin{enumerate}
        \item $E(M)$ is an orthogonal projection for each $M \in \Borel(Y)$; $E(\emptyset) = 0$ and $E(Y) = I$;
        \item $E(M_1 \cap M_2) = E(M_1)E(M_2)$ for all $M_i \in B(Y)$;
        \item $E(M_1 \sqcup M_2) = E(M_1) + E(M_2)$ for all disjoint $M_i \in B(Y)$;
        \item For every $x \in H$, $E_{x, x} : M \mapsto \lrangle{E(M)x, x}$ is a Radon measure in $Y$.
    \end{enumerate}
\end{definition}
Throughout this section, $Y \subseteq \C$ denotes a fixed subset. To define integrals of measurable functions with respect to spectral measures, we assign to each $f \in \M(Y)$ a densely defined operator on $H$ which corresponds to ``integrating'' the function with respect to the spectral measure. We first establish this for bounded measurable functions.
\begin{lemma}\label{bdint}
    Let $E$ be a spectral measure on $Y$. Then for each $f \in \M_\infty(Y)$ there is a unique operator $\int f dE \in B(H)$ such that for each $x \in H$:
    \begin{enumerate}
        \item $\lrangle{\int f dEx, x} = \int f dE_{x, x}$;
        \item $\norm{\int f dE x}^2 = \int\abs{f}^2 dE_{x, x}$
    \end{enumerate}
    Further, the map $\psi_0 : f \mapsto \int f dE$ is a $*$-homomorphism $\psi_0 : \M_\infty(Y) \to B(H)$.
\end{lemma}
\begin{proof}
    $(1)$ forces our map to be given by $\sum_{j = 1}^n \lambda_j \chi_{M_j} \mapsto \sum_{j = 1}^n \lambda_j E(M_j)$ on simple functions, which is well-defined as the values of $\lrangle{T(x), x}$ determine $T$. $\psi_0$ is continuous as $\norm{\psi_0(f)x}^2 \leq \norm{f}^2_\infty E_{x, x}(Y) = \norm{f}_\infty^2\norm{x}^2$, so the previous formula extends from simple functions to $\M_\infty(Y)$ by density. The desired properties follow by checking on simple functions, and taking limits with weak-continuity for general $f \in \M_\infty(Y)$.
\end{proof}
We extend this to all unbounded operators and measurable functions $f$ by taking sequences in $\M_\infty(Y)$ which are $L^2(Y, E_{x, x})$ and pointwise convergent, and for fixed $f \in \M(Y)$ we write $Y_{f, n} = \set{y \in Y \mid \abs{f(y)} \leq n}$.
\begin{proposition}\label{unbdint}
    Let $E$ be a spectral measure on $Y$. Then for each $f \in \M(Y)$, there is a unique densely defined operator $\psi(f)$ in $H$ with domain
    $$
        D(\psi(f)) = \set{x \in H \ \bigg| \ \int_Y \abs{f}^2 dE_{x, x} < \infty}
    $$
    so that if $x \in D(\psi(f))$ and $(f_n)_{n \in \N}$ in $\M_\infty(Y)$ converges in $L^2(Y, E_{x, x})$ to $f$, $\lim_{n \to \infty}\psi_0(f_n)x = \psi(f)x$, and
    \begin{enumerate}
        \item $\lrangle{\psi(f)x, x} = \int f dE_{x, x}$;
        \item $\norm{\int f dE x}^2 = \int \abs{f}^2 dE_{x, x}$.
    \end{enumerate}
\end{proposition}
\begin{proof}
    For each $f \in \M(Y)$, $D(\psi(f))$ is a subspace as $E_{x + y, x + y}(M) \leq 2(E_{x, x}(M) + E_{y, y}(M))$ by Cauchy-Schwarz, and $E_{\lambda x, \lambda x}(M) = \abs{\lambda}^2E_{x, x}$; and dense as for $x \in H$, $R(E(Y_{f, n})) \subseteq D(\psi(f))$, and $\norm{x - E(Y_{f, n})x}^2 = E_{x, x}(Y \setminus Y_{f, n}) \to 0$. Showing $\psi$ is well-defined is similar to extending integrals from simple functions to Lebesgue measurable functions.We get $\norm{\psi(f)x}^2 = \int \abs{f}^2dE_{x, x}$ and $\lrangle{\psi(f)x, x} = \int f dE_{x, x}$ by taking limits with $\M_\infty(Y) \ni f_n \to f$, noting that $f \in L^1(Y, E_{x, x})$ since $E_{x, x}$ is a finite-valued measure.
\end{proof}
As in Lemma \ref{bdint}, we will write $\int f dE$ for $\psi(f)$ as above. This notion of an integral has the following properties:
\begin{proposition}\label{psiprops}
    Let $E$ be a spectral measure on $Y$, $f \in \M(Y)$ and $\psi(f) = \int f dE$ be as above. Then for $f, g \in \M(Y)$,
    \begin{enumerate}
        \item $\psi(f) + \psi(g) \subset \psi(f + g)$;
        \item $\psi(f)\psi(g) \subset \psi(fg)$, and $D(\psi(f)\psi(g)) = D(\psi(g)) \cap D(\psi(fg))$.
        \item $\psi(f)^* = \psi(\ol f)$, and $\psi(f)\psi(f^*) = \psi(\abs{f}^2) = \psi(f^*)\psi(f)$, and hence $\psi(f)$ is closed.
    \end{enumerate}
\end{proposition}
\begin{proof}
    $(1)$ follows as $L^2(Y, E_{x, x})$ is closed under addition. For $(2)$ we compute $\int\abs{f}^2 dE_{\psi(g)x, \psi(g)x} = \int\abs{fg}^2 dE_{x, x}$ by taking sequences in $\M_\infty(Y)$ converging to $f$ and $g$, and the first statement follows as $\norm{\psi(fg)x - \psi(f_ng)x} = \int\abs{f - f_n}^2dE_{\psi(g)x, \psi(g)x}$, so for $\M_\infty(Y) \ni f_n \to f$ in $L^2(Y, E_{x, x})$ we have $\psi(f)\psi(g)x = \lim_{n \to \infty}\psi(f_n g)x = \psi(fg)x$. For (3), $\psi(\ol f) \subset \psi(f)^*$ follows by again taking $\M_\infty(Y) \ni f_n \to f$. The reverse inclusion follows as for $y \in D(\psi(f)^*)$ and $f_n \defeq f\chi_{Y_{f, n}}$, we have $E(Y_n)\psi(f)^* = \psi\brac{\ol{f_n}}$, so $\int_{Y_{f, n}}\abs{f}^2 dE_{y, y} = \norm{\psi\brac{\ol {f_n}}y}^2 \leq \norm{\psi(f)^*y}^2$, so $y \in D(\psi(\ol f))$. The remaining equality follows as $D(\psi(\abs{f}^2)) \subseteq D(\psi(f))$ by Cauchy-Schwarz, and $\psi(f) = \psi(\ol{f})^*$ is closed.
\end{proof}
% Possibly make this into a definition.
The above proposition shows that if $D(\psi(g)) = H$, then $D(\psi(f)\psi(g)) = D(\psi(fg))$. We need another characterisation of $\sigma\brac{\int f dE}$ in terms of $f$. We use the following notion, well-defined as $\C$ has a countable basis.
\begin{definition}
    Let $E$ be a spectral measure on $Y$ and $f \in \M(Y)$. The \textbf{essential range $\essim(f)$ (with respect to $E$)} is such that $\essim_E(f)^c \subseteq \C$ is the largest open subset $O$ with $E(f^{-1}(O)) = 0$.
\end{definition} % Rewrite this
\begin{proposition}\label{essimspec} % Can possibly fudge a bit of this
    Let $E$ be a spectral measure. Then for any $f \in \M(Y)$,
    $$
        \sigma\brac{\int f dE} = \essim_E(f)
    $$
\end{proposition}
\begin{proof}
    The map $g : z \mapsto \frac{\chi_M(z)}{\lambda - f(z)}$ is in $\M_\infty(Y)$ for $z \not\in \essim_E(f)$, the equation $g \cdot (\lambda - f) = (\lambda - f) \cdot g = \chi_M$ gives $\psi(g)\brac{\lambda I - \int f dE} \subset \brac{\lambda I - \int f dE}\psi(g) = \psi(\chi_M) = I$, so $\lambda \in \rho\brac{\int f dE}$. 
    % Essential - $\psi$ has norm 1.
    
    For $Y_\lambda = f^{-1}(\set\lambda)$, any $x \in R(E(Y_\lambda))$ has $\psi(f)x = \lambda x$ as $f\chi_{Y_\lambda} = \lambda\chi_{Y_\lambda}$. Thus for $\lambda \in \essim_E(f)$, if $E(Y_\lambda) \neq 0$ we are done. If $E(Y_\lambda) = 0$, for $M_n = f^{-1}(B(\lambda, 1/n)) = \set{y \in Y \mid \abs{\lambda - f(y)} < 1/n}$, we have $E(M_n) \neq 0$. Choosing $x_n \in R(E(M_n))$ with $\norm{x_n} = 1$ gives $\norm{\brac{\lambda I - \int f dE} x_n} \leq \norm{\lambda - f}_\infty \leq 1/n$, so $\brac{\lambda I - \int f dE}x_n \to 0$. If $\lambda \in \rho\brac{\int f dE}$, then $\brac{\lambda I - \int f dE}^{-1} \in B(H)$, and so $x_n \to 0$. Since $\norm{x_n} = 1$ this is not possible, so $\lambda \in \sigma\brac{\int f dE}$.
\end{proof}
To use the correspondence in Proposition \ref{corresp}, we use the following result to reduce to the Cayley transform.
\begin{lemma}
    Let $Y, Z \subseteq \C$, $\varphi : Y \to Z$ be a homeomorphism, and $F : \Borel(Z) \to B(H)$ be a spectral measure on $Z$. Then $E = F \circ \varphi : \Borel(Y) \to B(H)$ given by $E(M) = F(\varphi(M))$ is a spectral measure on $Y$, satisfying
    $$
        \int_Y fdE = \int_Z f \circ \varphi^{-1}dF
    $$
\end{lemma}
\begin{proof}
    The spectral measure properties hold as $\varphi$ is a homeomorphism. Since $\int_Y \chi_M dE = E(M) = F(\varphi(M)) = \int_Z \chi_{\varphi(M)} dF = \int_Z \chi_M \circ \varphi^{-1} dF$, the claimed equality holds for all simple functions, hence all $f \in \M_\infty(Y)$ by density. To extend this to $f \in \M(Y)$ we show $D(\int_Y f dE) = D(\int_Z f \circ \varphi^{-1} dF)$, which follows as $\int_Y \abs{f}^2 dE_{x, x} = \int_Z \abs{f \circ \varphi^{-1}}^2 dF_{x, x}$ for any $f \in \M_\infty(Y)$, thus for any $f \in \M(Y)$ by the monotone convergence theorem.
\end{proof}

% Spectral measures (own section; possibly do both one after other, sketches with ; props 18.3, 18.7, 18.9 (mention $\lrangle{Ax, x}$ determining $A$), 20.8 - 20.10; pair 18.3 + 18.9 and 20.8, )

% [Spectrum is the essential range]

\section{The Spectral Theorem}
We are almost ready to prove the spectral theorem. We sketch the theorem in the case of normal bounded operators, since it will be insightful to the underlying structure.
\begin{theorem}\label{specnorm}
    Let $T \in B(H)$ be normal, and $\Psi : \M_\infty(\sigma(T)) \to B(H)$ be its functional calculus. Then $E : \Borel(\sigma(T)) \to B(H)$ given by $E(M) = \Psi(\chi_M)$ is the unique spectral measure on $\sigma(T)$ with $T = \int z dE$.
\end{theorem}
\begin{proof}
    To show each $E_{x, x}$ defines a Radon measure, for each $x \in H$, we check that $\hat\mu : f \mapsto \lrangle{\Psi(f)x, x}$ is a positive functional on $\M_\infty(X)$, so that $\hat\mu(\chi_M)$ is a Radon measure, and that $E_{x, x}(M) = \hat\mu(\chi_M)$.

    For uniqueness, any spectral measure $F$ with $T = \int z dF$ gives a $*$-homomorphism $f \mapsto \int f dF$, which agrees with $\Psi$ on polynomials in $z$ and $\ol z$. This extends to $C(\sigma(T))$ by Stone-Weierstrass, as $\sigma(T)$ is compact, then to $\chi_O$ for $O \subseteq \sigma(T)$ open, by taking $(g_n)$ in $C(\sigma(T))$ bounded with $g_n \to \chi_O$ pointwise and applying dominated convergence. 
\end{proof}
We also give a lemma relating the eigenvalue decomposition of a bounded normal operator to its spectral measure.
\begin{lemma}\label{bdnormaleigen}
    Let $T \in B(H)$ be normal, and $E$ be the spectral measure of $T$. Then $E(\set{\lambda})$ is orthogonal projection onto the eigenspace $H_\lambda \defeq N(\lambda I - T)$, and in particular $\lambda \in \sigma(T)$ is an eigenvalue if and only if $E(\set{\lambda}) \neq 0$.
\end{lemma}
The proof of this lemma comes down to showing that $R(E(\set\lambda))$ is the $\lambda$-eigenspace of $T$. We are now equipped to prove the spectral theorem, which we state explicitly.
\begin{theorem}[The spectral theorem for unbounded self-adjoint operators]
    Let $T$ be self-adjoint. Then there is a unique spectral measure $E$ on $\sigma(T)$ with $T = \int z dE$. Explicitly, $E(M) = \Psi'\brac{\chi_{\varphi(M)}}$, for $\Psi'$ the functional calculus of the Cayley transform $\U(T)$ of $T$, and $\varphi : \sigma(T) \to \sigma(\U(T)) \setminus \set1$ given by $z \mapsto \frac{i - z}{-i - z}$.
\end{theorem}
\begin{proof}
    First, $U \defeq \U(T)$ admits a spectral measure $F$ by Theorem \ref{specnorm}. As $I - U$ is injective, $F(\set{1}) = 0$ by the previous lemma, so $F$ restricts to a spectral measure on $Z \defeq \sigma(U) \setminus \set1$, with $\int_{\sigma(U)} g dF = \int_Z g dF$ for all $g \in \M(Z)$. The map $f : S^1 \setminus \set1 \to \R$, $f(z) = i\frac{1 + z}{1 - z}$ is a homeomorphism and real-valued, so $B \defeq \Psi'(f) = \int_Z f dF$ is self-adjoint. Applying $\Psi'$ to the equation $f(z)(1 - z) = i(1 + z)$ gives $B(I - U) = i(I + U)$. Thus $A = i(I + U)(I - U)^{-1} \subset B$, and since $A$ and $B$ are self-adjoint we get $A = A^* \supset B^* = B$.

    The formula given for $\varphi$ defines a homeomorphism $\R \to S^1 \setminus \set1$ with inverse $f$, and for $\varphi$ to be well-defined we must have $f(Z) = \sigma(T)$. We note $G = \C \setminus f(Z)$ is open and $F(f|_Z^{-1}(G)) = F(\emptyset) = 0$. If $G_0 \supsetneq G$ is open, then $G_0 \cap Y \neq \emptyset$, so $f|_Z^{-1}(G_0) \subseteq Z$ is non-empty and open in $Z$, so $F(f|_Z^{-1}(G_0)) \neq 0$. We thus have $f(Z) = \essim(f|_Z) = \sigma(T)$, so $f^{-1}$ has a well-defined restriction $\varphi : \sigma(T) \to Z = \sigma(\U(T)) \setminus \set1$. Defining $E = F \circ \varphi$ as above, we have that $T = \int_Z f dF = \int_Z \varphi^{-1} dF = \int_{\sigma(T)} z dF$.

    For uniqueness, if $E$ is any spectral measure on $\sigma(T)$ with $T = \int z dE$, for $g = f|_{\sigma(T)}^{-1} : \sigma(T) \to \C$, we have $g(t)(-i - t) = i - t$, and applying $\psi$ yields $\brac{\int g dE}(-iI - T) = iI - T$, so $\int g dE$ is the Cayley transform of $U$, whose spectral measure is uniquely determined, hence so is the spectral measure of $T$.
\end{proof}
% \begin{lemma}
%     Eigenvalues for bounded normal operators
% \end{lemma}
As in the spectral theorem for compact operators, we will reason about the eigenvalue decomposition of $T$. 
\begin{definition}
    Let $T$ be a self-adjoint operator, and $E$ its spectral measure, and let $E_\lambda \defeq E((-\infty, \lambda] \cap \sigma(T))$. Then $(E_\lambda)_{\lambda \in \R}$ is the \textbf{spectral resolution} of $E$.
\end{definition}
Since each $E_{x, x}$ is a finite measure on $\sigma(T)$, the spectral resolution determines the spectral measure $E$. Using the spectral resolution, we obtain the following description of the spectrum and eigenvalues of $T$.
% [[Explicit description]]
\begin{theorem}\label{eigenvaldecomp}
    Let $T$ be self-adjoint, and $E$ its spectral measure. Then for $\mu \in \R$,
    \begin{enumerate}
        \item $\mu \in \EV(T)$ if and only if $E(\set{\mu}) \neq 0$, or equivalently $E_\mu \neq E_{\mu^-}$, where $E_{\mu^-}(x) = \lim_{\lambda \to \mu^-}E_\lambda(x)$;
        \item If $\mu \in \EV(T)$, then $E(\set{\mu}) = E_\mu - E_{\mu^-}$ is orthogonal projection onto $N(\mu I - T)$;
        \item $\mu \in \rho(T)$ if and only if $\lambda \mapsto E_\lambda$ is constant in some neighbourhood of $\mu$.
    \end{enumerate}
\end{theorem}
\begin{proof}
    Let $F$ be the spectral measure of $\U(T)$. The proof of the spectral theorem gives $E(\set\mu) = F\brac{\set{\frac{i - \mu}{-i - \mu}}}$, and $N(\mu I - T) = N(\frac{i - \mu}{-i - \mu}I - \U(T))$. We have $E_{\mu^-} = E((-\infty, \lambda) \cap \sigma(T))$ and thus $E_{\mu} - E_{\mu^-} = E(\set\mu)$, so $(1)$ and $(2)$ reduce to applying Lemma \ref{bdnormaleigen} to $\U(T)$. For $(3)$, as $\sigma(T)$ is closed, there is $(\a, \b) \ni \mu$ disjoint from $\sigma(T)$, and $E_\lambda = E_\a$ for $\lambda \in (\a, \b)$. Conversely, if $\lambda \mapsto E_\lambda$ is constant on $J = (\a, \b) \ni \mu$, then $E(J) = 0$ for $g \in \M_\infty(\sigma(T))$ given by $t \mapsto \frac{\chi_{\R \setminus J}(t)}{\mu - t}$, we have $g(t)(\lambda - t) = \chi_{\R \setminus J}$, so by Proposition \ref{psiprops} we get $\brac{\int g dE}(\mu I - T) \subset (\mu I - T)\brac{\int g dE} = E(\R \setminus J) = I$.
\end{proof}
The fact that our results for complex Hilbert spaces still involve purely real spectra should suggest that an analogous result should hold for real Hilbert spaces. We have also dealt exclusively with real Hilbert spaces, so such a result would tie in well with the results we have already seen.
\begin{theorem}
    Let $H$ be a real Hilbert space, and $T$ be a self-adjoint operator on $H$. Then there is a unique spectral measure $E$ with $T = \int \lambda dE$.
\end{theorem}
\begin{proof}
    Let $H_\C$ and $T_\C$ be the complexifications of $H$ and $T$ respectively, with $D(T_\C) = D(T) + iD(T)$. Then $T_\C$ is also self-adjoint, and $\sigma(A) = \sigma(A_\C)$ as $\sigma(A_\C) \subseteq \R$. Let $U \in B(H_\C)$ be the conjugation map $x + iy \mapsto x - iy$. Letting $E$ be the spectral measure of $T_\C$, $F : \Borel(\sigma(A_\C)) \to B(H_\C)$ with $F(M) = UE(M)U^{-1}$ is another spectral measure with $T = \int \lambda dF$, so by uniqueness we have $F = E$. This gives $E(M)(x) \in H$ for any $x \in H$, so $E_\R(M) = E(M)|_H$ is orthogonal projection in $H$, so $E_\R$ is a spectral measure on $\sigma(A) = \sigma(A_\C)$ with values in $B(H)$. Uniqueness follows from the uniqueness of $E$.
\end{proof} % Possibly change notation for $U$
We hence also obtain a eigenvalue decomposition as in Theorem \ref{eigenvaldecomp} for self-adjoint operators on real Hilbert spaces.


\section{Semigroups Generated by Self-Adjoint Operators}
We consider applications of the above theory to solutions of initial value problems. Consider the initial value problem $(*)$ given by $u'(t) = Tu(t)$ for $t \in [0, T]$ and $u(0) = u_0$. If $T$ is bounded and normal, we can consider the exponential $e^{tT}$ using the functional calculus. We use the spectral theorem to extend this to all self-adjoint operators. In light of viewing the solution as an exponential, we have the following notion:
\begin{definition}
    A family of bounded operators $S = \set{S(t)}_{t \geq 0}$ on $H$ is a $C_0$-semigroup if
    \begin{enumerate}
        \item (Multiplicative) $S(0) = I$ and $S(s + t) = S(s)S(t)$ for all $s, t \geq 0$;
        \item (Strong continuity) For all $x \in H$, $\lim_{t \to 0}\norm{S(t)x - x} = 0$.
    \end{enumerate} 
    The \emph{generator} of a $C_0$-semigroup $S$ is the operator $A_S$ with domain $D(A_S) = \set{x \in H \mid \lim_{t \to 0}\frac1t(S(t)x - x) \text{ exists}}$ and $A_Sx = \lim_{t \to 0}\frac1t(S(t)x - x)$.
\end{definition}
For a $C_0$-semigroup $S$, the function $u(t) = S(t)u_0$ is morally the solution to the initial value problem $(*)$, with $T = A_S$. This is justified by deeper theory which is not relevant for our purposes, but can be found in Proposition 13.4 and section 13.3 of \hyperlink{jvn}{[3]}. We consider semigroups generated by a self-adjoint operator, as our spectral theorem applies to these. We require the following proposition.
\begin{proposition}
    Every weakly-continuous semigroup is strongly continuous.
\end{proposition}
The idea is that $H_0 = \set{x \in X \mid \lim_{t \to 0} \norm{S(t)x - x} \to 0}$ is closed by uniform boundedness, and for an appropriate notion of an integral on $H$, $x_t \defeq \frac1t\int_0^tS(s)xds$ is well-defined and in $X_0$ for any $x \in H$ by weak-continuity, and $x_t \xrightarrow{w} x$, so $x \in H_0^{\perp\perp} = H_0$. We obtain the following result about semigroups generated by self-adjoint operators, whose statement explicitly requires the spectral theorem.
\begin{theorem}
    Let $T$ be self-adjoint with spectral measure $E$, and $\sigma(T) \subseteq [0, \infty)$. Then $-T$ generates a $C_0$-semigroup $S$ with $\sup\norm{S(t)} = 1$, given by
    $$
        S(t) = \int_{\sigma(T)} e^{-\lambda t} dE
    $$
\end{theorem}
\begin{proof}
    Multiplicativity of $S$ and the norm bound follow by properties of the functional calculus. We get $\lrangle{S(t)x, x} \xrightarrow{t \to 0} \lrangle{x, x}$ by dominated convergence, and polarisation and the above proposition give strong continuity. $T = A_S$ follows by a similar argument.
\end{proof}
We consider an application of this to the Laplacian, and a more thorough treatment of the basic spectral properties of the Laplacian can be found in Example 10.39 of \hyperlink{jvn}{[3]}. Let $\Delta = -\sum_{k = 1}^n \frac{\partial^2}{\partial x_i^2}$ be the Laplacian on $L^2(\R^n)$, initially defined on $C^2(\R^n)$. This admits a natural self-adjoint extension to $H^2(\R^n)$ by taking weak-partial derivatives, and we can compute its spectrum to be $[0, \infty)$ by noting that it corresponds to multiplication by $\norm{x}^2$ under the Fourier transform. The associated initial value problem $u'(t) = \Delta u(t)$ is exactly the heat equation, and by the above theorem, letting $E_\Delta$ be the spectral measure of $\Delta$, we find that the associated heat semigroup $H$ is given by
$$
    H(t) = \int_0^\infty e^{-\lambda t}dE_\Delta
$$
Thus we get an explicit form for the solutions of the heat equation on $\R^n$.
% The spectral theorem

% [[The normal case? (18.10; 18.14) -- maybe sketch]]

% [[Cayley transformations -- bijection; pullbacks -- may be better in spec meas section]]



% Eigenvalue decomposition (prop 18.14 + 20.15)

% The real case:

% Define complexifications; involution gives another spectral measure -- they agree, so we get a well-defined restriction to $H$.

\section{References}


% References:

% Meise-Vogt
[1] \hypertarget{meisevogt}{\textsc{R. Meise and D. Vogt}, (1997), \emph{Introduction to Functional Analysis}, Oxford University Press, Oxford, 198 - 230.}

[2] \hypertarget{rudin}{\textsc{W. Rudin}, (1991), \emph{Functional Analysis}, McGraw-Hill, New Jersey.}

[3] \hypertarget{jvn}{\textsc{J. van Neerven}, (2022), \emph{Functional Analysis}, Cambridge University Press, Cambridge.}

% jan van neerven
\end{document}